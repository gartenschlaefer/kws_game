% --
% ipa

\section{International Phonetic Alphabet}\label{sec:appendix_ipa}
The International Phonetic Alphabet (IPA) defines phonetics by human speaking sounds, where each symbol represents one specific sound.
A word formed with letters from a natural language alphabet, does not necessarily represent the pronunciation of that word, therefore many dictionaries provide an IPA transcription so that no misconceptions may happen.
The plots, in some sections within this thesis, contain phonetic transcriptions with IPA characters, some of the more special ones are described in \rtab{appendix_ipa}.

% ipa table
\input{./8_appendix/tables/tab_appendix_ipa.tex}

% --
% mathematical notations

\section{Mathematical Notations}\label{sec:appendix_math}
The mathematical equations and formulations are following a unified criteria.
Vectors and scalars are usually written in small letters with vectors marked as bold symbols.
In general vectors are assumed to be column vectors within this thesis.
Capital Letters are often representing matrices or transformed signals, but not necessarily.
Also capital letters are often used for fixed integer numbers, for instance the length of a signal window $N$.
The dimension of vectors and matrices are normally provided in the text, such as $\bm{x} \in \R^n$ or $X \in \C^{M \times N}$, otherwise it follows from the context.
Some symbols like $n$ or $x$ are likewise used in different sections, but with a different meaning and representations and should hopefully not confuse the reader of this thesis.
The logarithm notation $\log x$ without subscription is always referred to the natural logarithm with basis $e$.