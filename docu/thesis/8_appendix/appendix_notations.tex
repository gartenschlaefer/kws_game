% --
% ipa

\section{International Phonetic Alphabet}\label{sec:intro_overview_ipa}
The International Phonetic Alphabet (IPA) defines phonetics by human speaking sounds, where each symbol represents one specific sound.
A word formed with letters from a natural language alphabet, does not necessarily represent the pronunciation of that word, therefore many dictionaries provide an IPA transcription so that no misconceptions may happen.
The plots, in some sections within this thesis, contain phonetic transcriptions with IPA characters, some of the more special ones are described in \rtab{intro_overview_ipa}.

% ipa table
\begin{table}[ht!]
\begin{center}
\caption{Some IPA symbols and the silence symbol with phonetic descriptions.}
\begin{tabular}{ M{2cm}  M{11cm} }
\toprule
\textbf{IPA Symbol} & \textbf{Meaning} \\
\midrule
\textturnv & back vowel: \enquote{A}, open-mid roundend mouth \\
\textupsilon & back vowel: between \enquote{O} and \enquote{U}, nearly closed rounded mouth\\
\textinvglotstop & glottal stop\\
\midrule
sil & silence, no ipa symbol!\\
\bottomrule
\label{tab:intro_overview_ipa}
\end{tabular}
\end{center}
\vspace{-4mm}
\end{table}
\FloatBarrier
\noindent



% --
% mathematical notations

\section{Mathematical Notations}\label{sec:intro_overview_math}
The mathematical equations or expressions are following a unified criteria.
Vectors and scalars are usually written in small letters with no special indication for vectors.
%The from a vectors in row or column vector is depended on the application, for instance the position (left or right hand side to a matrix) for a matrix-vector multiplication.
In general vectors are assumed to be column vectors within this thesis.
%However in general the vectors are notated as column vectors, but the reader should not be confused when a matrix-vector multiplication is written as $x A$ (row vector notation) instead of $x^T A$ or $A x$ (column vector notation).
Capital Letters are often representing matrices or transformed signals, but not necessarily.
Also capital letters are often used for fixed integer numbers, for instance the length of a signal window $N$.
The dimension of vectors and matrices are normally provided in the text, such as $x \in \R^n$ or $X \in \C^{M \times N}$, or follow from the context.
Many letters like $n$ or $x$ are likewise used in different sections, but with different meaning and representations and should hopefully not confuse the reader of this thesis.
Regarding the logarithm notation, a $\log x$ without subscription is always referred as natural logarithm to the basis $e$.
%The letters $m$ and $n$ usually describes the length of a signal, $x$  and $y$ often represents input and output variables.