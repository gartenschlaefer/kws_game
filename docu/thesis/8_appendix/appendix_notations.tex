% --
% ipa

\section{International Phonetic Alphabet}\label{sec:appendix_ipa}
The International Phonetic Alphabet (IPA) are symbols that describe the pronouncing of phonemes.
A word formed with letters from a natural language alphabet, does not necessarily represent the pronunciation of that word, therefore many dictionaries provide an IPA transcription in order to avoid misconceptions in pronunciation.
The plots in some sections within this thesis contain transcriptions with IPA characters.
Some of the more special ones are described in \rtab{appendix_ipa}.

% ipa table
\begin{table}[ht!]
\small
\begin{center}
\caption{IPA symbols and the silence symbol with phonetic descriptions.}
\begin{tabular}{ M{2cm}  M{11cm} }
\toprule
\textbf{IPA Symbol} & \textbf{Meaning} \\
\midrule
\textturnv & back vowel: \enquote{A}, open-mid roundend mouth \\
\textupsilon & back vowel: between \enquote{O} and \enquote{U}, nearly closed rounded mouth\\
\textinvglotstop & glottal stop\\
\midrule
sil & silence, no ipa symbol!\\
\bottomrule
\label{tab:appendix_ipa}
\end{tabular}
\end{center}
\vspace{-4mm}
\end{table}
\FloatBarrier
\noindent



% --
% mathematical notations

\section{Mathematical Notations}\label{sec:appendix_math}
Vectors and scalars are usually written in small letters with vectors marked as bold symbols.
In general mathematical convention, the vectors are assumed to be column vectors.
Capital Letters are often representing matrices or fixed integer numbers, for instance the length of a signal window $N$.
The dimension of vectors and matrices are usually provided next to their formulation, such as $\bm{x} \in \R^n$ or $X \in \C^{M \times N}$, otherwise it follows from the context.
Some symbols like $n$ or $x$ are likewise used in different sections but with different meaning and representation and should hopefully not confuse the reader of this thesis.
The logarithm notation $\log x$ without subscription is always referred to the natural logarithm of basis $e$.