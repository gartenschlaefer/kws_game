% --
% dataset

\section{Dataset}\label{sec:exp_dataset}
Two datasets are used within this thesis, one is the speech commands dataset from \cite{Warden2018} and one is self made, denoted here as \enquote{my dataset} consisting only of 5 labels especially valuable for movement in video games (\enquote{left}, \enquote{right}, ...).
Note that the \enquote{my dataset} is merely used for evaluation and training of the neural networks is done on the speech commands dataset.
Both datasets consists of raw wav files, no feature extraction was done beforehand.
As already mentioned in \rsec{prev_kws_benchmark} direct comparisons between different neural network approaches is a bit difficult.
Some datasets provide feature extraction beforehand, such that the comparability of neural network architectures performances is not influenced on the data preparation.
More details of the datasets are presented below.

%This is very convenient, since the feature extraction is completely up to the users.
%It is therefore very important how features are extracted by the individual users and that direct comparisons between different approaches is a bit difficult.
%such that a direct comparability of scores between two independent researchers using different methods in feature extraction is quite difficult.



% Some abbreviations and references were done, so that the jungle of selected parameters get a little bit more clear to the reader of this thesis.
% The abbreviations of the dataset are shown in \rtab{exp_dataset_abbr}.

% The speech commands dataset is extracted before it is used for training. 
% To reduce computations in the evaluation process of neural networks, it was important to reduce the number of classes and examples per class to an suitable number.

% \input{./5_exp/tables/tab_exp_dataset_abbr.tex}
