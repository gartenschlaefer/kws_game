% --
% feature selection

\section{MFCC Feature Selection}\label{sec:exp_fs}
\thesisStateNotReady
Feature selection is a very important step prior to neural network training.
Unfortunately it is not always clear, which feature are contributing to good classification scores and which not.
To reduce features always means to reduce computations and is therefore a crucial point to evaulate.
As already described in \rsec{signal_mfcc}, Mel Frequency Cepstral Coefficients (MFCC) are extracted with a certain amount of filter bands and cepstral coefficients.
Further MFCC features can be enhanced with energy features and deltas.
A frame based normalization can be applied to improve the visualization of MFCCs and makes it more easy and faster to train Generative Adversarial Neural Networks (GANs).
However frame based normalization might be critical, because it focuses on the time scale and not the frequency. 

Two experiments are done to shade some light into MFCC feature selection:
\begin{enumerate}
    \item Impact on the amount of cepstral coefficients
    \item Impact on the enhancements of MFCCs
\end{enumerate}


\subsection{Impact on the Amount of Cepstral Coefficients}
The filter bands of the MFCCs are fixed with a total number of 32 and the cepstral coefficients are selected to either 12 or 32, where 12 is the common selection of coefficients in many papers.
Also the experiment is done once with and once without frame based normalization.
...

The experiment is shown in \rtab{exp_fs_cepstral}.
\begin{table}[ht!]
\small
\begin{center}
\caption{Experiment on the impact of the amount of cepstral coefficient of MFCC features. Frame based normalization was evaluated additionally.}
\begin{tabular}{ M{3cm}  M{2cm}  M{2cm}  M{2.5cm}  M{2.5cm} }
\toprule
\textbf{arch} & \textbf{mfcc} & \textbf{norm} & \textbf{acc test} & \textbf{acc my} \\
\midrule
conv-jim & mfcc32-12 & 0 & $83.60 \pm 2.35$ & $84.00 \pm 8.39$ \\
conv-jim & mfcc32-32 & 0 & $82.86 \pm 1.83$ & $82.40 \pm 5.43$ \\
conv-jim & mfcc32-12 & 1 & $77.72 \pm 0.96$ & $87.20 \pm 6.40$ \\
conv-jim & mfcc32-32 & 1 & $75.03 \pm 2.35$ & $83.20 \pm 4.66$ \\
\bottomrule
\label{tab:exp_fs_cepstral}
\end{tabular}
\end{center}
\vspace{-4mm}
\end{table}
\FloatBarrier
\noindent


% These enhancements (deltas and energy features) are formed in groups for evaluation to see the impact on the choice and hopefully to reduce the input feature size to a minimum.
% %Now that the neural network architectures are described in \rsec{nn_arch} and basic knowledge about MFCCs is given in \rsec{features} it is important to evaluate the impact of the selection of certain MFCC feature constellations to the accuracy of the Test sets.
% Beside it is good to get a general overview on what accuracies can be expected from different neural network architectures.
% The evaluation is done on 5 classes and 30 classes with different training parameters to observe the impact on a easy and a very hard classification task.
% In detail it is shown how models are trained with features consisting of following MFCC groups:
% \begin{enumerate}
%     \item Cepstral Coefficients (usual MFCCs)
%     \item Deltas (frame difference of MFCCs)
%     \item Double Deltas (frame difference of Deltas)
%     \item Energy Vector (added to each of the upper features)
% \end{enumerate}
% Another crucial point is to evaluate whether a frame based normalization of these features hurt the training and the accuracy of the models.
% Therefore additional columns are presented in the following tables marked with \enquote{norm}.
% Note that all these experiments have been done with n-500 a number of 500 examples per class, so that computations are minimized but still enough data is drawn.

\subsection{Feature Selection on Conv Encoder}
The feature selection evaluation on the conv-encoder-fc1 architecture with 5 labels is listed in \rtab{exp_fs_fc1_it500_c5}.
% \input{5_exp/tables/b1_feature_selection/ml_it500_c5_features_fc1}
% \input{5_exp/tables/b1_feature_selection/ml_it1000_c30_features_fc1}
% \input{5_exp/tables/b1_feature_selection/ml_it2000_c30_features_fc3}
\begin{table}[ht!]
\begin{center}
\caption{Feature Selection on arch: conv-encoder-fc1 with dataset: L5-n500 and training params: it500-bs32-lr0.0001-mo0.5}
\begin{tabular}{ M{1cm}  M{1cm}  M{1cm}  M{1cm}  M{1.5cm}  M{1.5cm}  M{1.5cm}  M{1.5cm} }
\toprule
\multicolumn{4}{c}{\textbf{Feature Groups}} & \multicolumn{2}{c}{\textbf{Accuracy}} \\
\textbf{c} & \textbf{d} & \textbf{dd} & \textbf{e} & \textbf{acc test} & \textbf{acc my} & \textbf{acc test norm} & \textbf{acc my norm} \\
\midrule
0 & 0 & 1 & 0 & 86.67 & 80.00 & 68.33 & 73.33 \\
0 & 0 & 1 & 1 & 85.00 & 86.67 & 67.67 & 73.33 \\
0 & 1 & 0 & 0 & 92.67 & 100.00 & 75.67 & 80.00 \\
0 & 1 & 0 & 1 & 90.67 & 90.00 & 82.00 & 73.33 \\
0 & 1 & 1 & 0 & 91.00 & 93.33 & 76.67 & 70.00 \\
0 & 1 & 1 & 1 & 89.33 & 100.00 & 78.67 & 80.00 \\
1 & 0 & 0 & 0 & 16.67 & 16.67 & 88.33 & 86.67 \\
1 & 0 & 0 & 1 & 33.33 & 33.33 & 86.33 & 80.00 \\
1 & 0 & 1 & 0 & 91.00 & 90.00 & 87.00 & 80.00 \\
1 & 0 & 1 & 1 & 82.67 & 86.67 & 86.67 & 90.00 \\
1 & 1 & 0 & 0 & 91.67 & 76.67 & 88.33 & 90.00 \\
1 & 1 & 0 & 1 & 90.00 & 80.00 & 89.33 & 93.33 \\
1 & 1 & 1 & 0 & 89.00 & 76.67 & 89.33 & 90.00 \\
1 & 1 & 1 & 1 & 88.00 & 90.00 & 89.00 & 86.67 \\
\bottomrule
\label{tab:exp_fs_fc1_it500_c5}
\end{tabular}
\end{center}
\vspace{-4mm}
\end{table}
\FloatBarrier
\noindent


\begin{table}[ht!]
\begin{center}
\caption{Feature Selection on arch: conv-encoder-fc1 with dataset: L30-n500 and training params: it1000-bs128-lr0.0001-mo0.5}
\begin{tabular}{ M{1cm}  M{1cm}  M{1cm}  M{1cm}  M{1.5cm}  M{1.5cm} }
\toprule
\multicolumn{4}{c}{\textbf{Feature Groups}} & \multicolumn{2}{c}{\textbf{Accuracy}} \\
\textbf{c} & \textbf{d} & \textbf{dd} & \textbf{e} & \textbf{acc test} & \textbf{acc test norm} \\
\midrule
0 & 0 & 1 & 0 & 52.97 & 33.10 \\
0 & 0 & 1 & 1 & 56.65 & 27.94 \\
0 & 1 & 0 & 0 & 63.55 & 39.68 \\
0 & 1 & 0 & 1 & 74.52 & 48.65 \\
0 & 1 & 1 & 0 & 71.03 & 39.35 \\
0 & 1 & 1 & 1 & 73.94 & 46.39 \\
1 & 0 & 0 & 0 & 47.23 & 58.19 \\
1 & 0 & 0 & 1 & 45.35 & 57.81 \\
1 & 0 & 1 & 0 & 60.45 & 47.55 \\
1 & 0 & 1 & 1 & 59.48 & 51.61 \\
1 & 1 & 0 & 0 & 58.77 & 55.10 \\
1 & 1 & 0 & 1 & 6.45 & 47.81 \\
1 & 1 & 1 & 0 & 65.81 & 60.39 \\
1 & 1 & 1 & 1 & 62.06 & 51.55 \\
\bottomrule
\label{tab:exp_fs_fc1_it1000_c30}
\end{tabular}
\end{center}
\vspace{-4mm}
\end{table}
\FloatBarrier
\noindent


\begin{table}[ht!]
\begin{center}
\caption{Feature Selection on arch: conv-encoder-fc3 with dataset: L30-n500 and training params: it2000-bs128-lr0.0001-mo0.5}
\begin{tabular}{ M{1cm}  M{1cm}  M{1cm}  M{1cm}  M{1.5cm}  M{1.5cm} }
\toprule
\multicolumn{4}{c}{\textbf{Feature Groups}} & \multicolumn{2}{c}{\textbf{Accuracy}} \\
\textbf{c} & \textbf{d} & \textbf{dd} & \textbf{e} & \textbf{acc test} & \textbf{acc test norm} \\
\midrule
0 & 0 & 1 & 0 & 49.03 & 33.74 \\
0 & 0 & 1 & 1 & 66.90 & 34.84 \\
0 & 1 & 0 & 0 & 77.10 & 55.55 \\
0 & 1 & 0 & 1 & 78.65 & 56.52 \\
0 & 1 & 1 & 0 & 74.97 & 50.26 \\
0 & 1 & 1 & 1 & 73.94 & 59.61 \\
1 & 0 & 0 & 0 & 76.52 & 64.26 \\
1 & 0 & 0 & 1 & 73.87 & 59.16 \\
1 & 0 & 1 & 0 & 78.58 & 63.03 \\
1 & 0 & 1 & 1 & 73.48 & 59.03 \\
1 & 1 & 0 & 0 & 79.10 & 66.13 \\
1 & 1 & 0 & 1 & 80.39 & 60.77 \\
1 & 1 & 1 & 0 & 76.97 & 64.71 \\
1 & 1 & 1 & 1 & 75.94 & 65.55 \\
\bottomrule
\label{tab:exp_fs_fc3_it2000_c30}
\end{tabular}
\end{center}
\vspace{-4mm}
\end{table}
\FloatBarrier
\noindent



\subsection{Feature Selection on fstride}
fstride
\begin{table}[ht!]
\begin{center}
\caption{Feature Selection on arch: conv-fstride with dataset: L5-n500 and training params: it1000-bs32-lr0.0001-mo0.5}
\begin{tabular}{ M{1cm}  M{1cm}  M{1cm}  M{1cm}  M{1.5cm}  M{1.5cm}  M{1.5cm}  M{1.5cm} }
\toprule
\multicolumn{4}{c}{\textbf{Feature Groups}} & \multicolumn{2}{c}{\textbf{Accuracy}} \\
\textbf{c} & \textbf{d} & \textbf{dd} & \textbf{e} & \textbf{acc test} & \textbf{acc my} & \textbf{acc test norm} & \textbf{acc my norm} \\
\midrule
0 & 0 & 1 & 0 & 65.67 & 56.67 & 43.67 & 53.33 \\
0 & 0 & 1 & 1 & 64.33 & 60.00 & 49.00 & 43.33 \\
0 & 1 & 0 & 0 & 86.00 & 83.33 & 69.00 & 63.33 \\
0 & 1 & 0 & 1 & 85.00 & 76.67 & 67.67 & 83.33 \\
0 & 1 & 1 & 0 & 84.67 & 76.67 & 72.33 & 73.33 \\
0 & 1 & 1 & 1 & 84.00 & 80.00 & 76.00 & 66.67 \\
1 & 0 & 0 & 0 & 88.33 & 76.67 & 84.33 & 73.33 \\
1 & 0 & 0 & 1 & 90.00 & 86.67 & 81.67 & 70.00 \\
1 & 0 & 1 & 0 & 89.00 & 80.00 & 86.00 & 86.67 \\
1 & 0 & 1 & 1 & 88.67 & 83.33 & 83.67 & 80.00 \\
1 & 1 & 0 & 0 & 89.33 & 83.33 & 86.33 & 73.33 \\
1 & 1 & 0 & 1 & 90.00 & 80.00 & 86.67 & 80.00 \\
1 & 1 & 1 & 0 & 89.00 & 76.67 & 86.33 & 73.33 \\
1 & 1 & 1 & 1 & 90.67 & 86.67 & 86.00 & 86.67 \\
\bottomrule
\label{tab:exp_fs_fstride_it1000_c5}
\end{tabular}
\end{center}
\vspace{-4mm}
\end{table}
\FloatBarrier
\noindent



\subsection{Feature Selection on trad}
trad
\begin{table}[ht!]
\begin{center}
\caption{Feature Selection on arch: conv-trad with dataset: L5-n500 and training params: it1000-bs32-lr0.0001-mo0.5}
\begin{tabular}{ M{1cm}  M{1cm}  M{1cm}  M{1cm}  M{1.5cm}  M{1.5cm}  M{1.5cm}  M{1.5cm} }
\toprule
\multicolumn{4}{c}{\textbf{Feature Groups}} & \multicolumn{2}{c}{\textbf{Accuracy}} \\
\textbf{c} & \textbf{d} & \textbf{dd} & \textbf{e} & \textbf{acc test} & \textbf{acc my} & \textbf{acc test norm} & \textbf{acc my norm} \\
\midrule
0 & 0 & 1 & 0 & 84.00 & 86.67 & 65.33 & 56.67 \\
0 & 0 & 1 & 1 & 85.67 & 86.67 & 72.00 & 66.67 \\
0 & 1 & 0 & 0 & 91.67 & 83.33 & 84.67 & 66.67 \\
0 & 1 & 0 & 1 & 93.33 & 93.33 & 83.33 & 80.00 \\
0 & 1 & 1 & 0 & 93.33 & 86.67 & 86.33 & 80.00 \\
0 & 1 & 1 & 1 & 91.67 & 90.00 & 90.33 & 80.00 \\
1 & 0 & 0 & 0 & 94.33 & 93.33 & 92.00 & 83.33 \\
1 & 0 & 0 & 1 & 95.00 & 90.00 & 91.00 & 93.33 \\
1 & 0 & 1 & 0 & 93.67 & 90.00 & 94.67 & 83.33 \\
1 & 0 & 1 & 1 & 91.67 & 96.67 & 94.00 & 76.67 \\
1 & 1 & 0 & 0 & 95.00 & 83.33 & 92.67 & 83.33 \\
1 & 1 & 0 & 1 & 94.67 & 93.33 & 92.00 & 86.67 \\
1 & 1 & 1 & 0 & 95.33 & 90.00 & 92.00 & 86.67 \\
1 & 1 & 1 & 1 & 95.00 & 100.00 & 94.00 & 86.67 \\
\bottomrule
\label{tab:exp_fs_trad_it1000_c5}
\end{tabular}
\end{center}
\vspace{-4mm}
\end{table}
\FloatBarrier
\noindent



