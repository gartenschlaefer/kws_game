% --
% feature selection

\section{Feature Selection}\label{sec:exp_fs}
The first important Question, when using neural networks, is what features are used as inputs.
In the feature extraction section about MFCCs \rsec{signal_mfcc}, it was shown how raw audio files can be extracted to MFCCs and what enhancements can be done.
These enhancements (deltas and energy features) are formed in groups for evaluation to see the impact on the choice and hopefully to reduce the input feature size to a minimum.
%Now that the neural network architectures are described in \rsec{nn_arch} and basic knowledge about MFCCs is given in \rsec{features} it is important to evaluate the impact of the selection of certain MFCC feature constellations to the accuracy of the Test sets.
Beside it is good to get a general overview on what accuracies can be expected from different neural network architectures.
The evaluation is done on 5 classes and 30 classes with different training parameters to observe the impact on a easy and a very hard classification task.
In detail it is shown how models are trained with features consisting of following MFCC groups:
\begin{enumerate}
    \item Cepstral Coefficients (usual MFCCs)
    \item Deltas (frame difference of MFCCs)
    \item Double Deltas (frame difference of Deltas)
    \item Energy Vector (added to each of the upper features)
\end{enumerate}
Another crucial point is to evaluate whether a frame based normalization of these features hurt the training and the accuracy of the models.
Therefore additional columns are presented in the following tables marked with \enquote{norm}.
Note that all these experiments have been done with n-500 a number of 500 examples per class, so that computations are minimized but still enough data is drawn.

\subsection{Feature Selection on Conv Encoder}
The feature selection evaluation on the conv-encoder-fc1 architecture with 5 labels is listed in \rtab{exp_fs_fc1_it500_c5}.
% \begin{table}[ht!]
\begin{center}
\caption{Feature Selection ml it500 c5 features fc1}
\begin{tabular}{ M{1cm}  M{1cm}  M{1cm}  M{1cm}  M{1.5cm}  M{1.5cm}  M{1.5cm}  M{1.5cm} }
\toprule
\multicolumn{4}{c}{\textbf{Feature Groups}} & \multicolumn{2}{c}{\textbf{Accuracy}} \\
\textbf{c} & \textbf{d} & \textbf{dd} & \textbf{e} & \textbf{acc test} & \textbf{acc my} & \textbf{acc test norm} & \textbf{acc my norm} \\
\midrule
0 & 0 & 1 & 0 & 86.67 & 80.00 & 68.33 & 73.33 \\
0 & 0 & 1 & 1 & 85.00 & 86.67 & 67.67 & 73.33 \\
0 & 1 & 0 & 0 & 92.67 & 100.00 & 75.67 & 80.00 \\
0 & 1 & 0 & 1 & 90.67 & 90.00 & 82.00 & 73.33 \\
0 & 1 & 1 & 0 & 91.00 & 93.33 & 76.67 & 70.00 \\
0 & 1 & 1 & 1 & 89.33 & 100.00 & 78.67 & 80.00 \\
1 & 0 & 0 & 0 & 16.67 & 16.67 & 88.33 & 86.67 \\
1 & 0 & 0 & 1 & 33.33 & 33.33 & 86.33 & 80.00 \\
1 & 0 & 1 & 0 & 91.00 & 90.00 & 87.00 & 80.00 \\
1 & 0 & 1 & 1 & 82.67 & 86.67 & 86.67 & 90.00 \\
1 & 1 & 0 & 0 & 91.67 & 76.67 & 88.33 & 90.00 \\
1 & 1 & 0 & 1 & 90.00 & 80.00 & 89.33 & 93.33 \\
1 & 1 & 1 & 0 & 89.00 & 76.67 & 89.33 & 90.00 \\
1 & 1 & 1 & 1 & 88.00 & 90.00 & 89.00 & 86.67 \\
\bottomrule
\end{tabular}
\end{center}
\label{tab:ml_it500_c5_features_fc1}
\end{table}
\FloatBarrier
\noindent


% \input{5_exp/tables/b1_feature_selection/ml_it1000_c30_features_fc1}
% \input{5_exp/tables/b1_feature_selection/ml_it2000_c30_features_fc3}
\begin{table}[ht!]
\begin{center}
\caption{Feature Selection on arch: conv-encoder-fc1 with dataset: L5-n500 and training params: it500-bs32-lr0.0001-mo0.5}
\begin{tabular}{ M{1cm}  M{1cm}  M{1cm}  M{1cm}  M{1.5cm}  M{1.5cm}  M{1.5cm}  M{1.5cm} }
\toprule
\multicolumn{4}{c}{\textbf{Feature Groups}} & \multicolumn{2}{c}{\textbf{Accuracy}} \\
\textbf{c} & \textbf{d} & \textbf{dd} & \textbf{e} & \textbf{acc test} & \textbf{acc my} & \textbf{acc test norm} & \textbf{acc my norm} \\
\midrule
0 & 0 & 1 & 0 & 86.67 & 80.00 & 68.33 & 73.33 \\
0 & 0 & 1 & 1 & 85.00 & 86.67 & 67.67 & 73.33 \\
0 & 1 & 0 & 0 & 92.67 & 100.00 & 75.67 & 80.00 \\
0 & 1 & 0 & 1 & 90.67 & 90.00 & 82.00 & 73.33 \\
0 & 1 & 1 & 0 & 91.00 & 93.33 & 76.67 & 70.00 \\
0 & 1 & 1 & 1 & 89.33 & 100.00 & 78.67 & 80.00 \\
1 & 0 & 0 & 0 & 16.67 & 16.67 & 88.33 & 86.67 \\
1 & 0 & 0 & 1 & 33.33 & 33.33 & 86.33 & 80.00 \\
1 & 0 & 1 & 0 & 91.00 & 90.00 & 87.00 & 80.00 \\
1 & 0 & 1 & 1 & 82.67 & 86.67 & 86.67 & 90.00 \\
1 & 1 & 0 & 0 & 91.67 & 76.67 & 88.33 & 90.00 \\
1 & 1 & 0 & 1 & 90.00 & 80.00 & 89.33 & 93.33 \\
1 & 1 & 1 & 0 & 89.00 & 76.67 & 89.33 & 90.00 \\
1 & 1 & 1 & 1 & 88.00 & 90.00 & 89.00 & 86.67 \\
\bottomrule
\label{tab:exp_fs_fc1_it500_c5}
\end{tabular}
\end{center}
\vspace{-4mm}
\end{table}
\FloatBarrier
\noindent


\input{./5_exp/tables/tab_exp_fs_fc1_it1000_c30.tex}
\begin{table}[ht!]
\begin{center}
\caption{Feature Selection on arch: conv-encoder-fc3 with dataset: L30-n500 and training params: it2000-bs128-lr0.0001-mo0.5}
\begin{tabular}{ M{1cm}  M{1cm}  M{1cm}  M{1cm}  M{1.5cm}  M{1.5cm} }
\toprule
\multicolumn{4}{c}{\textbf{Feature Groups}} & \multicolumn{2}{c}{\textbf{Accuracy}} \\
\textbf{c} & \textbf{d} & \textbf{dd} & \textbf{e} & \textbf{acc test} & \textbf{acc test norm} \\
\midrule
0 & 0 & 1 & 0 & 49.03 & 33.74 \\
0 & 0 & 1 & 1 & 66.90 & 34.84 \\
0 & 1 & 0 & 0 & 77.10 & 55.55 \\
0 & 1 & 0 & 1 & 78.65 & 56.52 \\
0 & 1 & 1 & 0 & 74.97 & 50.26 \\
0 & 1 & 1 & 1 & 73.94 & 59.61 \\
1 & 0 & 0 & 0 & 76.52 & 64.26 \\
1 & 0 & 0 & 1 & 73.87 & 59.16 \\
1 & 0 & 1 & 0 & 78.58 & 63.03 \\
1 & 0 & 1 & 1 & 73.48 & 59.03 \\
1 & 1 & 0 & 0 & 79.10 & 66.13 \\
1 & 1 & 0 & 1 & 80.39 & 60.77 \\
1 & 1 & 1 & 0 & 76.97 & 64.71 \\
1 & 1 & 1 & 1 & 75.94 & 65.55 \\
\bottomrule
\label{tab:exp_fs_fc3_it2000_c30}
\end{tabular}
\end{center}
\end{table}
\FloatBarrier
\noindent



\subsection{Feature Selection on fstride}
fstride
\begin{table}[ht!]
\begin{center}
\caption{Feature Selection on arch: conv-fstride with dataset: L5-n500 and training params: it1000-bs32-lr0.0001-mo0.5}
\begin{tabular}{ M{1cm}  M{1cm}  M{1cm}  M{1cm}  M{1.5cm}  M{1.5cm}  M{1.5cm}  M{1.5cm} }
\toprule
\multicolumn{4}{c}{\textbf{Feature Groups}} & \multicolumn{2}{c}{\textbf{Accuracy}} \\
\textbf{c} & \textbf{d} & \textbf{dd} & \textbf{e} & \textbf{acc test} & \textbf{acc my} & \textbf{acc test norm} & \textbf{acc my norm} \\
\midrule
0 & 0 & 1 & 0 & 65.67 & 56.67 & 43.67 & 53.33 \\
0 & 0 & 1 & 1 & 64.33 & 60.00 & 49.00 & 43.33 \\
0 & 1 & 0 & 0 & 86.00 & 83.33 & 69.00 & 63.33 \\
0 & 1 & 0 & 1 & 85.00 & 76.67 & 67.67 & 83.33 \\
0 & 1 & 1 & 0 & 84.67 & 76.67 & 72.33 & 73.33 \\
0 & 1 & 1 & 1 & 84.00 & 80.00 & 76.00 & 66.67 \\
1 & 0 & 0 & 0 & 88.33 & 76.67 & 84.33 & 73.33 \\
1 & 0 & 0 & 1 & 90.00 & 86.67 & 81.67 & 70.00 \\
1 & 0 & 1 & 0 & 89.00 & 80.00 & 86.00 & 86.67 \\
1 & 0 & 1 & 1 & 88.67 & 83.33 & 83.67 & 80.00 \\
1 & 1 & 0 & 0 & 89.33 & 83.33 & 86.33 & 73.33 \\
1 & 1 & 0 & 1 & 90.00 & 80.00 & 86.67 & 80.00 \\
1 & 1 & 1 & 0 & 89.00 & 76.67 & 86.33 & 73.33 \\
1 & 1 & 1 & 1 & 90.67 & 86.67 & 86.00 & 86.67 \\
\bottomrule
\label{tab:exp_fs_fstride_it1000_c5}
\end{tabular}
\end{center}
\vspace{-4mm}
\end{table}
\FloatBarrier
\noindent



\subsection{Feature Selection on trad}
trad
\begin{table}[ht!]
\begin{center}
\caption{Feature Selection on arch: conv-trad with dataset: L5-n500 and training params: it1000-bs32-lr0.0001-mo0.5}
\begin{tabular}{ M{1cm}  M{1cm}  M{1cm}  M{1cm}  M{1.5cm}  M{1.5cm}  M{1.5cm}  M{1.5cm} }
\toprule
\multicolumn{4}{c}{\textbf{Feature Groups}} & \multicolumn{2}{c}{\textbf{Accuracy}} \\
\textbf{c} & \textbf{d} & \textbf{dd} & \textbf{e} & \textbf{acc test} & \textbf{acc my} & \textbf{acc test norm} & \textbf{acc my norm} \\
\midrule
0 & 0 & 1 & 0 & 84.00 & 86.67 & 65.33 & 56.67 \\
0 & 0 & 1 & 1 & 85.67 & 86.67 & 72.00 & 66.67 \\
0 & 1 & 0 & 0 & 91.67 & 83.33 & 84.67 & 66.67 \\
0 & 1 & 0 & 1 & 93.33 & 93.33 & 83.33 & 80.00 \\
0 & 1 & 1 & 0 & 93.33 & 86.67 & 86.33 & 80.00 \\
0 & 1 & 1 & 1 & 91.67 & 90.00 & 90.33 & 80.00 \\
1 & 0 & 0 & 0 & 94.33 & 93.33 & 92.00 & 83.33 \\
1 & 0 & 0 & 1 & 95.00 & 90.00 & 91.00 & 93.33 \\
1 & 0 & 1 & 0 & 93.67 & 90.00 & 94.67 & 83.33 \\
1 & 0 & 1 & 1 & 91.67 & 96.67 & 94.00 & 76.67 \\
1 & 1 & 0 & 0 & 95.00 & 83.33 & 92.67 & 83.33 \\
1 & 1 & 0 & 1 & 94.67 & 93.33 & 92.00 & 86.67 \\
1 & 1 & 1 & 0 & 95.33 & 90.00 & 92.00 & 86.67 \\
1 & 1 & 1 & 1 & 95.00 & 100.00 & 94.00 & 86.67 \\
\bottomrule
\label{tab:exp_fs_trad_it1000_c5}
\end{tabular}
\end{center}
\end{table}
\FloatBarrier
\noindent



