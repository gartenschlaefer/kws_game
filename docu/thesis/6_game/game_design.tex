% --
% game interactables

\section{Game Design}\label{sec:game_design}
\thesisStateNotReady
In this section the game design for the deployed KWS video game is presented.
Prior to explaining the game and its rules, it is important to mention the game menu and its vital settings for the input microphone device.
The game rules are explaining the win and loose condition of the video game.
Game mechanics restrict how the player can interact in the world and how to create actions for changing the states of game objects.

% --
% menu

\subsection{Menu}\label{sec:game_design_menu}
The game menu is the first screen that appears, when starting a video game. 
It usually consists of selectable buttons referencing for instance to the game settings of graphic and music.
With speech input deployed in a video game, the setting of the recording device is useful to be added.
All available microphone input devices should be visualized in a list and being selectable for being able to switch between the independent devices.
Further a small visualization bar of the input signal energy from the selected device is beneficial, so that the user can verify its correct functionality.
An option for adjusting the energy threshold, by which the device detects the onset of a speech signal, must be added if this property is used within the video game.
The threshold must be adjustable because of varying recording amplification factors in different microphone set ups.


% --
% game rules

\subsection{Game Rules}\label{sec:game_design_rules}
The game rules were hold simple, so that the game part does not take too much effort to implement.
A simple adventure game in a 2D-platformer view (movement are left-right and jump, like the classical Super Mario games), were the player has to collect object and dodge enemies was implemented.
The win condition is to complete each individual level by collecting a single object within each level and the loose condition is to touch the enemy.
If the player runs into the enemy, a loose screen appears and the same level can be restarted from its initial state.
Being able to win the game, the player has to use speech commands, otherwise the objects cannot be reached.
Further descriptions are presented below in the game mechanics and the level design.


% --
% game mechanics

\subsection{Game Mechanics}\label{game_design_mechanics}
The game mechanics with KWS are a highly time restricted, because of the processing time of the speech signals.
In this thesis the KWS system was used as augmented input control.
With standard keys for movement on the keyboard, the controlling of the character is handled, so that the player receives immediate feedback. 
For other more special actions, that do not require immediate feedback, a KWS control was applied.

The game mechanic implemented in the deployed video game, is to move a single block in two dimensions by saying the intended direction for instance \enquote{left} and \enquote{right} or \enquote{up} and \enquote{down}.
For being able to move more than one of those so-called movable block, the player can say \enquote{go} to switch between the blocks to the desired one, that should be moved.
At all times, therefore only the selected movable block of all moveable block can be controlled by the player.


% --
% enemy

\subsection{Enemy}\label{sec:game_design_enemy}
Being able to provide a loose condition within the game, enemies were implemented.
If the enemy touches the player, the game is over and the actual level has to be restarted.
The player can simply jump over the enemy to dodge.
The movement of the enemy is simple to run from left to right and if an obstacle is hit (like a wall) the enemy turns around and goes in the opposite direction.


% --
% level design

\subsection{Level Design}\label{sec:game_design_level}
Two levels were implemented with the game mechanic of movable blocks, as described previously in \rsec{game_design_mechanics}.
The first level requires the player to learn how the game works and to move blocks out of the players path to the collectable object.
In the second level the player has to align the moveable blocks, such that by jumping upon them, higher plateaus can be reached.
The levels are both shown in ...



