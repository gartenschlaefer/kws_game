% --
% abstract

\chapter*{Abstract}\label{sec:shards_abstract}
\thesisStateReady
Key Word Spotting (KWS) is a valuable tool in the interaction between humans and machines and can be applied in various situations.
This thesis specializes on the domain of video games and evaluates possible neural network architectures for speech command classification tasks.
%In focus are Convolutional Neural Network (CNN) architectures with adversarial pre-training using Mel Frequency Cepstral Coefficients (MFCC) as input features and Wavenets applied on raw audio samples.
In focus are Convolutional Neural Network (CNN) architectures using Mel Frequency Cepstral Coefficients (MFCC) as input features and Wavenets applied on raw audio samples.
An adversarial pre-training method is evaluated, that uses convolutional weights of Generative Adversarial Neural Networks (GAN). trained in a separate training instance, for CNN networks with the same convolutional layer structure.
The energy consumption and efficiency is of utter importance in video games and therefore a strong criterion within this thesis.
A simple online system for KWS is implemented in a video game to evaluate the game experience of using speech commands as actions within the game.

%Further some possible video game ideas or scenarios with speech command inputs with the purpose of triggering events within the games are evaluated and discussed.


% --
% german abstract

\chapter*{Kurzfassung}
\thesisStateReady
Key Word Spotting (KWS) ist ein wertvolles Werkzeug in der Mensch-Maschine Kommunikation und kann in vielen unterschiedlichen Situationen angewendet werden.
Diese Arbeit ist auf Video Spiele spezialisiert und evaluiert mögliche Neural Network Architekturen für Sprachbefehls-Erkennung.
Der Fokus liegt bei Convolutional Neural Network (CNN) Architekturen die Mel Frequency Cepstral Coefficients (MFCC) als input features verwenden und Wavenets, welche auf rohen audio daten operieren.
...
