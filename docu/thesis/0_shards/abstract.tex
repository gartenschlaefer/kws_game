% --
% abstract

\chapter*{Abstract}\label{sec:shards_abstract}
\thesisStateReady
Key Word Spotting (KWS) is a valuable tool in the interaction between humans and machines and can be applied in various situations and aspects.
This thesis specializes on the domain of video games and evaluates possible neural network architectures for speech command classification tasks.
In focus are Convolutional Neural Network (CNN) architectures using Mel Frequency Cepstral Coefficients (MFCC) as input features and Wavenets applied on raw audio samples.
An adversarial pre-training method for CNNs is evaluated, that uses the weights of the convolutional layers of Generative Adversarial Neural Networks (GAN) with the same convolutional layer structure as the CNN network and trained in a separate training instance.
The energy consumption and efficiency is of utter importance in video games and therefore a strong criterion within this thesis.
A simple KWS system is implemented in a video game to evaluate the game experience of using speech commands as actions within the game.


% --
% german abstract

\chapter*{Kurzfassung}
\thesisStateReady
Key Word Spotting (KWS) ist ein wertvolles Werkzeug in der Interaktion zwischen Mensch und Maschine und kann in vielen unterschiedlichen Situationen und Bereichen angewendet werden.
Diese Arbeit bezieht sich insbesondere auf Computerspiele und evaluiert verschiedene Neurale Netzwerk Architekturen für Sprachbefehlserkennung.
Der Fokus liegt vor allem bei Convolutional Neural Network (CNN) Architekturen, welche Mel Frequency Cepstral Coefficients (MFCC) als Eingangs-Features verwenden und auf Wavenets, die auf rohen Audio Daten operieren.
Eine Methode des Adversarial Vortrainierens von CNNs wird evaluiert, die die Gewichte der Convolutional Layers eines Generative Adversarial Networks (GAN) verwendet welche in einem separaten Trainingsdurchlauf trainiert wurden, dabei soll das CNN Netzwerk dieselbe Convolutional Layer Struktur, wie das CNN aufweisen.
Der Energieverbrauch und die Effizienz ist äußerst bedeutsam in der Anwendung von Computerspielen und erhält daher auch in dieser Arbeit einen hohen Stellenwert.
Ein einfaches KWS System wurde in einem Computerspielen implementiert und die Spielerfahrung bei der Verwendung von Sprachbefehlen evaluiert.