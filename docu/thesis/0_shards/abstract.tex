% --
% abstract

\chapter*{Abstract}\label{sec:shards_abstract}
Keyword Spotting (KWS) is a valuable tool in the speech interaction between humans and machines and can be applied in various situations.
This thesis specifically considers the domain of video games and evaluates possible neural network architectures for speech command classification tasks.
In focus are Convolutional Neural Network (CNN) and Wavenet architectures.
The used input features are Mel Frequency Cepstral Coefficients (MFCCs) and raw audio samples for the CNN and Wavenet models, respectively.
Furthermore, this work evaluates a pre-training method that initializes the convolutional layers of a CNN by transferring the filter weights from the convolutional layers of a separately trained Generative Adversarial Neural Network (GAN).
In the ideal case, the computational tasks in video games should be very efficiently implemented such that the fluidity of the game does not suffer during computationally demanding phases while playing.
This also holds for KWS systems deployed in video games.
Therefore, this thesis concentrates on finding a neural network model that provides a low computational footprint for KWS tasks in video games.
Finally, this work presents a specifically designed video game with an integrated KWS system, where the player is able to move objects by using speech commands.


% --
% german abstract

\begin{otherlanguage}{ngerman}
\chapter*{Kurzfassung}
\enquote{Keyword Spotting (KWS)} ist ein wertvolles Werkzeug in der Sprach-Interaktion zwischen Mensch und Maschine und findet in vielen Bereichen Anwendung.
Diese Arbeit befasst sich insbesondere mit der Sprachbefehlserkennung in Computerspielen und evaluiert verschiedene neuronale Netzwerke für diese Aufgabe.
Das Hauptaugenmerk liegt vor allem auf \enquote{Convolutional Neural Network (CNN)}- und Wavenet-Architekturen.
Hierbei werden für die CNN- und Wavenet-Modelle, \enquote{Mel Frequency Cepstral Coefficients (MFCCs)} beziehungsweise rohe Audiodaten als Eingangs-Features verwendet.
Des Weiteren wird in dieser Arbeit eine Methode zur Initialisierung der \enquote{Convolutional Layers} eines CNNs mit den \enquote{Convolutional Layers} eines separat trainierten \enquote{Generative Adversarial Neural Network (GAN)} evaluiert.
Idealerweise sollen Berechnungsaufgaben in Computerspielen sehr effizient implementiert sein, um eine flüssige Spielbarkeit während rechenintensiven Phasen gewährleisten zu können.
Das gilt auch für KWS-Systeme, die in Computerspielen eingesetzt werden.
Daher ist ein wesentliches Ziel dieser Arbeit, ein neuronales Netzwerk mit geringer Rechenlast für KWS in Computerspielen zu finden.
Abschließend wird in dieser Arbeit ein eigens entwickeltes Computerspiel mit integriertem KWS-System präsentiert, wobei der Spieler Objekte mit Hilfe von Sprachbefehlen bewegen kann.
\end{otherlanguage}