% --
% abstract

\chapter*{Abstract}\label{sec:shards_abstract}
Key Word Spotting (KWS) is a valuable tool in the interaction between humans and machines and can be applied in various situations and aspects.
This thesis specializes in the domain of video games and evaluates possible neural network architectures for speech command classification tasks.
In focus are Convolutional Neural Network (CNN) and Wavenet architectures, where the used input features are Mel Frequency Cepstral Coefficients (MFCC) and raw audio samples for the CNN and Wavenet models respectively.
Moreover, an adversarial pre-training method evaluates the initialization of the convolutional layers of a CNN with the convolutional layers of a Generative Adversarial Neural Network (GAN), where the convolutional layer structure of the GAN models and the CNN model has to coincide.
The computational tasks in video games should ideally be implemented very efficiently such that the fluidity of the game does not suffer during high performance peaks.
A strong criterion within this thesis was therefore the determination of a neural network with a low computational footprint for the KWS task in video games.
A simple KWS system was implemented in a video game to evaluate the game experience of using speech commands as actions within the game.


% --
% german abstract

\chapter*{Kurzfassung}
Key Word Spotting (KWS) ist ein wertvolles Mittel zur Interaktion zwischen Mensch und Maschine und findet in vielen Situationen und Bereichen Anwendung.
Diese Arbeit bezieht sich insbesondere auf die Sprachbefehlserkennung in Computerspielen mit verschiedenen Neuronalen Netzwerken.
Der Fokus liegt vor allem bei Convolutional Neural Network (CNN) und Wavenet Architekturen, wobei für die CNN Modelle Mel Frequency Cepstral Coefficients (MFCC) und für die Wavenets rohe Audiodaten als Eingangs-Features verwendet werden.
Desweiteren wird eine Methode des adversarial pre-trainings zur Initialisierung der Convolutional Layers eines CNN mit den Convolutional Layers eines Generative Adversarial Neural Network (GAN) evaluiert, wobei die Struktur der Convolutional Layers der GAN Modelle und des CNN Models gleich sein muss.
Berechnungsaufgaben in Computerspielen sollten idealerweise sehr effizient implementiert sein, um eine flüssige Spielbarkeit während Spitzenleistungen gewährleisten zu können.
Daher war es ein wesentliches Ziel dieser Arbeit, ein Neurales Netzwerk mit geringer Berechnungsleistung für KWS in Computerspielen zu finden.
Ein einfaches KWS System wurde dazu in einem Computerspiel implementiert, um die Spielerfahrung bei der Verwendung von Sprachbefehlen zu evaluieren.