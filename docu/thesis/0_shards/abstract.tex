% --
% abstract

\chapter*{Abstract}\label{sec:shards_abstract}
Keyword Spotting (KWS) is a valuable tool in the interaction between humans and machines and can be applied in various situations and aspects.
This thesis specializes in the domain of video games and evaluates possible neural network architectures for speech command classification tasks.
In focus are Convolutional Neural Network (CNN) and Wavenet architectures.
The used input features are Mel Frequency Cepstral Coefficients (MFCC) and raw audio samples for the CNN and Wavenet models, respectively.
Furthermore, this work evaluates a pre-training method that initializes the convolutional layers of a CNN by transferring the filter weights from the convolutional layers of a separately trained Generative Adversarial Neural Network (GAN).
In the ideal case, the computational tasks in video games should be very efficiently implemented such that the fluidity of the game does not suffer during high performance peaks while playing.
This also holds for KWS systems used in video games.
Therefore, this thesis concentrates on finding a neural network that provides a low computational footprint.
Finally, this work presents a specifically designed video game with an integrated KWS system, where the player is able to move objects by using speech commands.


% --
% german abstract

\chapter*{Kurzfassung}
Keyword Spotting (KWS) ist ein wertvolles Mittel in der Interaktion zwischen Mensch und Maschine und findet in vielen Situationen und Bereichen Anwendung.
Diese Arbeit bezieht sich insbesondere auf Sprachbefehlserkennung in Computerspielen und evaluiert verschiedene neuronale Netzwerke für diese Aufgabe.
Das Hauptaugenmerk liegt vor allem auf Convolutional Neural Network (CNN) und Wavenet Architekturen.
Wobei für die CNN und Wavenet Modelle Mel Frequency Cepstral Coefficients (MFCC) beziehungsweise rohe Audiodaten als Eingangs-Features verwendet werden.
Des Weiteren wird in dieser Arbeit eine Methode zur Initialisierung der Convolutional Layers eines CNN mit den Convolutional Layers eines separat trainierten Generative Adversarial Neural Network (GAN) evaluiert.
Idealerweise sollen Berechnungsaufgaben in Computerspielen sehr effizient implementiert sein, um eine flüssige Spielbarkeit während Spitzenleistungen gewährleisten zu können.
Das gilt auch für KWS Systeme, welche in einem Computerspiel verwendet werden.
Daher ist ein wesentliches Ziel dieser Arbeit, ein neuronales Netzwerk mit geringer Berechnungsleistung zu finden.
Zum Schluss wird in dieser Arbeit ein eigens entwickelten Computerspiel mit integriertem KWS System präsentiert, wobei der Spieler Objekte mit Hilfe von Sprachbefehlen bewegen kann.