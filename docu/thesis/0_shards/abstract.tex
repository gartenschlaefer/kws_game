% --
% abstract

\chapter*{Abstract}\label{sec:shards_abstract}
Key Word Spotting (KWS) is a valuable tool in the interaction between humans and machines and can be applied in various situations and aspects.
This thesis specializes on the domain of video games and evaluates possible neural network architectures for speech command classification tasks.
In focus are Convolutional Neural Network (CNN) architectures using Mel Frequency Cepstral Coefficients (MFCC) as input features and Wavenets applied on raw audio samples.
An adversarial pre-training method evaluates the initialization of the convolutional layers of a CNN with the convolutional layers of a Generative Adversarial Neural Network (GAN), where the convolutional layer structure of the GAN models and the CNN model must coincide.
The processing of computational tasks in video games often require an efficient implementation of those to ensure the fluidity of the game during high performance peaks.
A strong criterion within this thesis is therefore the determination of a neural network with a low computational footprint for the KWS task in video games.
A simple KWS system was implemented in a video game to evaluate the game experience of using speech commands as actions within the game.


% --
% german abstract

\chapter*{Kurzfassung}
Key Word Spotting (KWS) ist ein wertvolles Mittel in der Interaktion zwischen Mensch und Maschine und findet in vielen Situationen und Bereichen Anwendung.
Diese Arbeit bezieht sich insbesondere auf die Sprachbefehlserkennung mit verschiedenen Neuronalen Netzwerken in Computerspielen.
Der Fokus liegt vor allem bei Convolutional Neural Network (CNN) Architekturen und Wavenets, wobei CNNs Mel Frequency Cepstral Coefficients (MFCC) als Eingangs-Features verwenden und Wavenets auf rohen Audio Daten angewendet werden.
Eine Methode des Adversarial Vortrainierens zur Initialisierung der Convolutional Layers eines CNN mit den Convolutional Layers eines Generative Adversarial Neural Network (GAN) wird evaluiert, wobei the Struktur der Convolutional Layers der GAN Modelle und des CNN Models gleich sein muss.
Die Abwicklung von Berechnungsaufgaben in Computerspielen benötigt oft eine effiziente Implementierung dieser, um eine flüssige Spielbarkeit bei Spitzenleistungen gewährleisten zu können.
Daher ist die ein starkes Kriterium dieser Arbeit ein Neurales Netzwerk mit geringer Berechnungsleistung für KWS in Computerspielen zu finden.
Ein einfaches KWS System wurde in einem Computerspiel implementiert, um die Spielerfahrung bei der Verwendung von Sprachbefehlen zu evaluieren.