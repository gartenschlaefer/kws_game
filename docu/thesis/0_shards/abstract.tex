% --
% abstract

\chapter*{Abstract}\label{sec:shards_abstract}
\thesisStateReady
Key Word Spotting (KWS) is a valuable tool in the interaction between humans and machines and can be applied in various situations.
This thesis specializes on the domain of video games and evaluates possible neural network architectures for speech command classification tasks.
%In focus are Convolutional Neural Network (CNN) architectures with adversarial pre-training using Mel Frequency Cepstral Coefficients (MFCC) as input features and Wavenets applied on raw audio samples.
In focus are Convolutional Neural Network (CNN) architectures using Mel Frequency Cepstral Coefficients (MFCC) as input features and Wavenets applied on raw audio samples.
An adversarial pre-training method is evaluated, that uses convolutional weights of Generative Adversarial Neural Networks (GAN) trained in a separate training instance, for CNN networks with the same convolutional layer structure.
The energy consumption and efficiency is of utter importance in video games and therefore a strong criterion within this thesis.
A simple KWS system is implemented in a video game to evaluate the game experience of using speech commands as actions within the game.

%Further some possible video game ideas or scenarios with speech command inputs with the purpose of triggering events within the games are evaluated and discussed.


% --
% german abstract

\chapter*{Kurzfassung}
\thesisStateReady
Key Word Spotting (KWS) ist ein wertvolles Werkzeug in der Mensch-Maschine Kommunikation und kann in vielen unterschiedlichen Situationen angewendet werden.
Diese Arbeit bezieht sich insbesondere auf Video Spiele und evaluiert verschiedene Neural Network Architekturen für Sprachbefehls-Erkennung.
Der Fokus liegt vor allem bei Convolutional Neural Network (CNN) Architekturen, die Mel Frequency Cepstral Coefficients (MFCC) als Eingangs Features verwenden und Wavenets, welche auf rohen audio daten operieren.
Eine Methode zum adversarial Vortrainieren der Convolutional Netzwerk Gewichte mit Generative Adversarial Networks (GAN) in einem eigenen Trainingsdurchlauf wird evaluiert, wobei das CNN Netzwerk die selbe convolutional Layer Struktur haben soll.
Der Energyverbrauch und die Effizienz is von besonderer Wichtigkeit bei der Anwendung in Computer Spiele und erhält daher auch in dieser Arbeit einen hohen Stellenwert.
Ein einfaches KWS System in einem Video Spiel wurde implementiert und die Spielerfahrung bei der Verwendung von Sprachbefehlen evaluiert.
