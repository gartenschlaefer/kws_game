% --
% prev key word spotting

\section{Key Word Spotting with Neural Networks}\label{sec:prev_kws}
\thesisStateRevised
Some important works regarding KWS with neural networks are presented here. 
The aspect of energy efficients is very important within this thesis, because of the deployment of a KWS system in a video game.
Further the neural network architectures evaluated in this thesis are trained and tested on an already profoundly examined speech command dataset \cite{Warden2018}.
Actual benchmarks are provided on classification accuracies from different neural network approaches.


% --
% energy efficient

\subsection{Energy efficient solutions}
In video games the processing and classification of speech commands has to run in real-time and therefore a low computational footprint is needed for the neural networks.
One of the most famous papers on low computational footprint regarding KWS systems is from Sainath et. al. in 2015 \cite{Sainath2015}.
Two neural network architectures are chosen from this paper, one is a traditional CNN network for comparison and the other is a limited multipliers network with a CNN striding only in the frequency axis.
Both networks are described in detail in \rsec{nn_arch}.

A deployment of a KWS system on microcontrollers is examined in \cite{Zhang2017}.
Different neural network architectures were evaluated regarding their memory usage and operations per inference.
The work showed that 32-bit to 8-bit quantization does not reduce the classification accuracy and that the depthwise separable CNN (DS-CNN) was the best choice with regards to limited amounts of memory and operations compared to the other evaluated networks.

Another highly optimized resource efficient Deep Neural Network (DNN) based on CNNs obtained with a Neural Architecture Search (NAS) is described by \cite{Peter2020}.
This work also showed that a 1-bit quantization of the weights performs nearly as good as the 8-bit quantization, which decreases the memory usage even further.


% --
% benchmark

\subsection{Benchmark Networks for this thesis}\label{sec:prev_kws_benchmark}
First it is to mention that the speech command dataset \cite{Warden2018} exists in two versions \texttt{v0.01} and \texttt{v0.02} consisting of raw audio data in the \texttt{.wav} format and there is no pre-processing or feature extraction done beforehand.
Also it is up to the users for which labels are to be selected.
The main idea however is to choose the \emph{core words} as classification labels and add an \enquote{unknown} label for the \emph{auxiliary words} words.
Also a separate noise or \enquote{silence} label can be added from given noise data files.
In most papers the core words were \{\enquote{left},  \enquote{right}, \enquote{up}, \enquote{down}, \enquote{go}\, \enquote{stop}, \enquote{yes}, \enquote{no}, \enquote{on}, \enquote{off}\} and two label for \enquote{silence} and \enquote{unknown} (all other words) used for comparison to previous papers.
This constellation of selected key words is referred here as \texttt{L12}, the speech command dataset version is referenced with \texttt{V1} or \texttt{V2}.
More details about the dataset is presented in \rsec{exp_dataset}.

One benchmark is hold by \cite{Berg2021} introducing a Key Word Transformer (KWT) network, which performs accuracy scores of almost 100\%.
A good overview of actual benchmark scores regarding the speech command dataset is given in \cite{PaperswithcodeKWS}.

All benchmark scores from the referenced works in this section are shown in \rtab{prev_kws_bench}.
\begin{table}[ht!]
\small
\begin{center}
\caption{Benchmark on the speech commands dataset.}
\begin{tabular}{ M{4cm} M{2cm} M{2cm} M{2.5cm} }
\toprule
\multirow{2}{*}{\centering\textbf{Model Name}} & \multicolumn{2}{c}{\textbf{Accuracy}} & \multirow{2}{*}{\centering\textbf{Operations}} \\
& V1-L12 & V2-L12 & \\
\midrule
DS-CNN-S \cite{Zhang2017} & 94.4 & - & \SI{5.4}{\mega\ops} \\
DNN-MFCC20 \cite{Peter2020} & 96.3 & - & \SI{27.1}{\mega\ops}\\
KWT-3 \cite{Berg2021} & 97.2 & 98.5 & ? \\
\bottomrule
\label{tab:prev_kws_bench}
\end{tabular}
\end{center}
\vspace{-4mm}
\end{table}
\FloatBarrier
\noindent
Note that this scores are achieved by very sophisticated training techniques and neural network architectures, incorporating many layers.
It is very hard to achieve the same performance as presented in those papers.
%For this thesis the benchmarks listed above, are rather
%that this thesis merely works on the foundations of KWS with basic neural network models, therefore those scores are very unlikely to be reached.

