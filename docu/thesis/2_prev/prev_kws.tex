% --
% prev keyword spotting

\section{Work on Keyword Spotting with Neural Networks}\label{sec:prev_kws}
The presented work on KWS with neural networks were selected to computationally efficient solutions and benchmark networks for the speech command dataset \cite{Warden2018}.


% --
% energy efficient

\subsection{Energy Efficient Solutions}
The processing and classification of speech commands for video games have to run in real-time and therefore a low computational footprint is required for the neural networks embedded in the classification system.
One famous paper on low computational footprint of CNNs regarding KWS tasks, was presented by Sainath et al. in 2015 \cite{Sainath2015}.
Two neural network architectures were chosen from this paper, one is a traditional CNN network and the other a limited multipliers CNN striding only in the frequency axis.
Both networks are described in detail in \rsec{nn_arch}.

The deployment of a KWS system on microcontrollers was examined in \cite{Zhang2017}.
Different neural network architectures were evaluated regarding their memory usage and operations per inference.
The work showed that 32-bit to 8-bit quantization does not reduce the classification accuracy and that the depthwise separable CNN (DS-CNN) was the best choice with regards to limited amounts of memory and operations in comparison to the other evaluated models.

Another highly optimized resource efficient Deep Neural Network (DNN) based on CNNs with parameters obtained by a Neural Architecture Search (NAS), is described by \cite{Peter2020}.
This work also showed that a 1-bit quantization of the weights perform nearly as well as the 8-bit quantization, which decreases the memory usage even further.


% --
% benchmark

\subsection{Benchmark Networks for this Thesis}\label{sec:prev_kws_benchmark}
First it is to mention that the speech command dataset \cite{Warden2018} exists in two versions (\texttt{v0.01} and \texttt{v0.02}) consisting of raw audio data in the \texttt{.wav} format and there is no pre-processing or feature extraction done beforehand.
Also it is up to the users for which labels are to be selected.
The intended concept is to choose the \emph{core words} as classification labels and add an \enquote{unknown} label for the \emph{auxiliary words}.
Also a separate noise or \enquote{silence} label can be added from given noise data files.
In most papers the core words were \{\enquote{left},  \enquote{right}, \enquote{up}, \enquote{down}, \enquote{go}, \enquote{stop}, \enquote{yes}, \enquote{no}, \enquote{on}, \enquote{off}\} and additional two labels for \enquote{silence} and \enquote{unknown} (all other words) used for comparison to previous papers.
This constellation of selected keywords is referred here as \texttt{L12} and the speech command dataset version is referenced by \texttt{V1} or \texttt{V2}.
More details about the dataset is presented in \rsec{exp_dataset}.

One benchmark is hold by \cite{Berg2021} introducing a Keyword Transformer (KWT) network, which performs accuracy scores of almost \SI{100}{\percent}.
A quick overview of continually updated benchmark scores regarding the speech command dataset, is provided in \cite{PaperswithcodeKWS}.

All benchmark scores from the referenced work in this section are shown in \rtab{prev_kws_bench}.
\begin{table}[ht!]
\small
\begin{center}
\caption{Benchmark on the speech commands dataset.}
\begin{tabular}{ M{4cm} M{2cm} M{2cm} M{2.5cm} }
\toprule
\multirow{2}{*}{\centering\textbf{Model Name}} & \multicolumn{2}{c}{\textbf{Accuracy}} & \multirow{2}{*}{\centering\textbf{Operations}} \\
& V1-L12 & V2-L12 & \\
\midrule
DS-CNN-S \cite{Zhang2017} & 94.4 & - & \SI{5.4}{\mega\ops} \\
DNN-MFCC20 \cite{Peter2020} & 96.3 & - & \SI{27.1}{\mega\ops}\\
KWT-3 \cite{Berg2021} & 97.2 & 98.5 & ? \\
\bottomrule
\label{tab:prev_kws_bench}
\end{tabular}
\end{center}
\vspace{-4mm}
\end{table}
\FloatBarrier
\noindent
Note that this scores are achieved by very sophisticated neural network architectures incorporating many layers.
It is very hard to achieve the same performance as presented in those papers.