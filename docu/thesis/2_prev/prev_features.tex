% --
% prev features

\section{Audio features for speech recognition}\label{sec:pref_features}
\thesisStateRevised
The extraction of features from audio signals is important for an efficient data compression of raw audio samples and a further processing or classification of those audio signals.
Speech signals are different from for instance musical signals.
In music the important features are rhythm and pitch, while speech does not use any of those specifically.
Speech can be spoken in different duration and frequency pitch, depending on the speaker.
Therefore speech signals are usually extracted to \emph{low-level features}, that do not incorporate specific structure of duration and pitch patterns.

One of the most popular low-level feature for speech recognition are the Mel Frequency Cepstral Coefficients (MFCC), introduced by \cite{Mermelstein1980} in 1980.
MFCCs are motivated by the physiological human hearing system using equidistant mel-frequency bands and are described in detail in \rsec{signal_mfcc}.
Another popular low-level feature similar to MFCCs are Perceptual Linear Prediction (PLP) features introduced by \cite{Hermansky1987}.
PLPs were not evaluated in this thesis, but it would have been interesting how they perform compared to MFCCs.
A good summary and comparison between PLP and MFCC can be found in \cite{Hoenig2005}.

Recently raw audio samples can be used as audio features for inputs to some special kinds of neural networks, such as the Wavenet architecture, more information is presented in the following sections.
