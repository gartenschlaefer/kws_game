% --
% prev games

\section{Video Games with Key Word Spotting}\label{sec:prev_kws_games}
\thesisStateReady
There are not many papers that are tackling Key Word Spotting (KWS) specifically tied to video games and most research is done separately to gain the best accuracies on datasets and setting new benchmark scores on them.
The only research paper found, was that of a multiplayer video game intended for children presented in \cite{Harshavardhan2015}, which evaluates challenges when children are playing video games, such as repetitive and overlapping commanding.
There are plenty of real world applications for KWS and Automatic Speech Recognition (ASR) using available toolkits.
One very powerful tool for speech commanding widely used for video games is \texttt{VoiceAttack} running on \enquote{Windows Speech Recognition} \cite{Xiong2017}.
With \texttt{VoiceAttack} it is possible to specify own voice commands that are triggering combinations of key board clicks and therefore can be used to elicit actions in video games.
The applications are vast and flexible and can be applied in any game by running the program in the background.
An actual video game with integrated speech commanding, done with the \texttt{PocketSphinx} \cite{Huggins2006} speech recognition system, is \texttt{In Verbis Virtus}, where the player is able to cast spells by speaking out fantasy words after clicking an activation button as onset indication.
The mentioned video games and applications can be looked up in an internet search for more detailed information upon them.