% --
% prev games

\section{Video Games with Keyword Spotting}\label{sec:prev_kws_games}
Very few papers are tackling KWS specifically tied to video games and most research is conducted on gaining the best accuracies on datasets and setting new benchmark scores on them.
Nevertheless, a research paper on multiplayer video games intended for children, is presented in \cite{Harshavardhan2015}, which evaluates challenges when children are playing video games, such as repetitive and overlapping commanding.
On the other hand, plenty of KWS and ASR toolkits for real world applications exist.
One very powerful ASR tool, especially tailored to use speech commands in video games, is \texttt{VoiceAttack} running on \enquote{Windows Speech Recognition} \cite{Xiong2017}.
With \texttt{VoiceAttack} it is possible to specify own voice commands that are triggering combinations of keyboard clicks and therefore can be used to elicit any possible action that the video game provides.
The applications are vast and flexible and can be applied in any video game by running the program in the background.
An actual video game with integrated speech commanding ability, running with the \texttt{PocketSphinx} \cite{Huggins2006} speech recognition system, is \texttt{In Verbis Virtus}, where the players are able to cast spells by speaking out fantasy words after clicking an activation button as onset indication.