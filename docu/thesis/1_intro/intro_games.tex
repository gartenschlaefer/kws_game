% --
% video games with speech commands

\section{Video Games with Keyword Spotting}\label{sec:intro_games}
Video games that use controls through KWS, or in general, through any speech input, are rarely seen gems in the gaming industry although KWS is an exciting and immersive tool to interact within video games.
Technically the voice of a player has to be recorded by a microphone, hence one additional requirement to play the video game is to own a microphone, which however does not need to be high-end.
The input stream of a microphone can be processed through an online or real-time system to extract the input features for the classification system.
The classification system, such as a neural network, is supposed to infer the input information to the intended action within the video game.

The processing scheme of a KWS system is easy to define but much harder to implement compared to hardware based input channels, such as pressing buttons on a keyboard.
Also speech input is very slow compared to hardware input, where players are able to interact within tens of milliseconds (the input lag of gaming controllers should ideally be under \SI{50}{\milli\second}).
Achieving equally fast response times is not possible with speech, where the player has to physically form a waveform representing the intended action.
Further, the waveform captured from the online system has to be pre-processed, feature extracted, and classified to the best estimation of all the available keywords in the game's vocabulary.
Concluding this, a good estimate to create and process speech input should ideally be under one second. 
The less the better for the playing experience.

Another important consideration has to be made upon the size of the vocabulary.
A high number of keywords will increase the chance of confusion and a small number of keywords constrains the possible actions a player can choose from.
The possibility to construct several separate classifiers with different sets of smaller vocabularies of speech commands arises.
The selection of the active classifier out of all classifiers would therefore depend on the commands a player requires in the actual scene of the video game.
For instance, if the player is in a dialogue with a Non Player Character (NPC), it is reasonable to reduce the vocabulary to only \{\enquote{yes}, \enquote{no}\}, if those are the only actions to choose from.

The use of labels for background noise, silence, and unknown words depends on the game and how the classification process is activated.
If the assumption is made that a player chooses only from the keywords in the dictionary, then the label of unknown words is not necessarily required.
Nevertheless, the unknown label would be very interesting when keywords are not that common to a player, such as words from a different language or fantasy words.
The unknown label will therefore motivate the player to correctly pronounce each individual word in the vocabulary, which could be very valuable in language learning games.
The labels for background noise and silence are useful in the case when the classification process is initialized by the energy value of the input stream from the microphone.
A small impact on the microphone or loud background sound may consequently elicit an unwanted command, if those labels do not exist in the vocabulary.

In KWS video games, players have to act with their voice, which is in some situations not appropriate or even annoying.
For instance, playing a video game in a public transport and shouting into ones mobile phone or laptop, might disturb other passengers.
Also the slowness of KWS and many other reasons makes it not very suitable for every game type.
Luckily the existing video game types and genres are vast and there might exist several reasonable game ideas, where KWS makes a game outstanding, creative, and refreshing.
Potential of KWS lies without doubt in the domain of Virtual Reality (VR) and Augmented Reality (AR) applications and games.
In VR, the players wear a Head Mounted Device (HMD), where most recent consumer devices have already integrated a microphone intended for online chat but also ready to adopt for KWS.
To increase immersion in VR even further, a KWS systems can be very valuable.