% --
% Intro to neural networks

\section{Neural Networks for Automatic Speech Recognition}\label{sec:intro_nn}

Neural Networks enable computers to automatically learn abstract representations of given data examples, for instance waveform files.
Those data examples can be paired with annotations, usually denoted as \emph{labels} or \emph{classes}.
If labels are used to \emph{train} a Machine Learning system, it is called \emph{supervised learning} otherwise it is called \emph{unsupervised learning}.

The big advantage of Neural Networks is that they are able to cope with a huge amount of input variables and are able to extract their own features of the inputs through many layers within the network.

In this thesis, the word \emph{feature} can have several meanings, one is just meant to be extracted data features and therefore be the same as input variables. 
Another is, that a feature is simply some kind of compressed representation of a high dimensional data.

Neural Networks are therefore able to learn their own feature representations, selection and interpretation, rather than using hand-crafted ones done by humans with expertise in the application. 

What sounds great at first hand, is also a bit of a downfall.
Mainly because everyone who is capable of using machine learning tools, can create a solution to a rather complex problem usually solved by experts, given there is enough data and processing power available.
This yield into less understanding of the actual problem and \enquote{try and error} approaches, since Neural Networks just need to be feed with the right amount of data and parameters for training.
This led to the thinking that everyone, who owns a lot of data and computational power, is the superior of solving complex problems such as image or audio classifications.

However it is not always like this rather negative case on using Neural Networks. 
It is extremely interesting to work with them and get knowlege about how and why they are able to produce such good results.
This feedbacks again experts to gain more understanding or a different viewpoint on the topic.
