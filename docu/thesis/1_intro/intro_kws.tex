% --
% Intro of key word spotting

\section{The Key Word Spotting Task}\label{sec:intro_kws}
As described in \rsec{intro}, KWS is the task of classifying speech signals of spoken words to single key words out of a set of key words.
The set of key words $S$, also denoted as vocabulary, can be defined as:
\begin{equation}\label{eq:intro_kws_dict}
	S \coloneqq \{s_i \mid i = 0, 1, \dots, L\}
\end{equation}
with a total number of $L$ key words contained in the vocabulary and individual key words $s_i$ indexed with $i$.
The task is to select the key word closest to the spoken word from the user, denoted as target $t$.
The target does not necessarily have to be a member of the set of key words $S$, in fact it can be any arbitrary word.
With the abstract formulation:
\begin{equation}\label{eq:intro_kws_task}
	\hat{s} = \underset{s_i \in S}{\arg \min} \, \mathcal{D}(t, s_i)
\end{equation}
the most probable key word $\hat{s}$ can be predicted, where $\mathcal{D}$ is some kind of distance measure between two words.
The formulation in \req{intro_kws_task} is merely semantic but KWS in computer systems must cope with various transformations of raw input samples of audio data denoted as $\bm{x} \in \R^n$ with a total number of $n$ samples.
An inference from audio data to output class probabilities $\bm{y} \in \R^L$ can, for example, be achieved by the use of a neural network containing a softmax function at its last layer (transforming the output of the last layer to probability values).
The most probable key word can therefore be picked by:
\begin{equation}\label{eq:intro_kws_class}
	\hat{s} = \{s_i \mid \underset{i = 0, 1, \dots, L}{\arg \max} \, y_i\}
\end{equation}
where the index $i$ of the output class probability $y_i$ must correspond to the intended key word $s_i$ in the vocabulary.

In comparison to full Automatic Speech Recognition (ASR), where whole sentences need to be identified, KWS operates merely on the word level.
Therefore, KWS is a bit easier to deploy and less complex than ASR.
On the other hand KWS systems must run very energy efficiently on low energy devices, such as mobile phones, and provide immediate and accurate responses to the users.
A good elaboration on the requirements of KWS systems can be found in the motivation section of \cite{Warden2018}.