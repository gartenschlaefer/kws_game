% --
% Intro of key word spotting

\section{The Key Word Spotting Task}\label{sec:intro_kws}
\thesisStateRevised
As described in \rsec{intro}, KWS is the task of classifying speech signals of spoken words to single key words out of a set of key words.
The set of key words $S$, also called vocabulary, can be defined as:
% kws dict
\begin{equation}\label{eq:intro_kws_dict}
	S \coloneqq \{s_i \mid i = 0, 1, \dots, L\}
\end{equation}
with a total number of $L$ key words denoted individually as $s_i$ for each key word.
The task is to select the key word closest to the spoken word from the user, denoted as target $t$.
The target does not necessarily have to be a member in the set of key words $S$, in fact it can be any arbitrary word.
With the abstract formulation:
% kws task
\begin{equation}\label{eq:intro_kws_task}
	\hat{s} = \underset{s_i \in S}{\arg \min} \, \mathcal{D}(t, s_i)
\end{equation}
the most probable key word $\hat{s}$ can be predicted, where $\mathcal{D}$ is some kind of distance measure between two words.
The formulation in \req{intro_kws_task} is merely semantic, but KWS in computer systems must cope with various transformations of raw input samples of audio data denoted as $\bm{x} \in \R^n$, with a total number of $n$ samples.
From the audio data an inference to output class probabilities $\bm{y} \in \R^L$ with a total number of $L$ class labels or key words, can be processed for instance with a neural network containing a softmax function at its last layer (transforming the output of the last layer to probability values).
With the softmax output from a neural network, the most probable key word can be picked by:
\begin{equation}\label{eq:intro_kws_class}
	\hat{s} = \{s_i \mid \underset{i = 0, 1, \dots, L}{\arg \max} \, y_i\}
\end{equation}
if the highest value of $y_i$ for all $i$ refers to the most probable index or indices of a spotted key word or key words in the vocabulary.

In comparison to full Automatic Speech Recognition (ASR), where whole sentences need to be identified, key word spotting operates merely on the word level.
Therefore KWS is a bit easier to deploy and less complex than ASR.
On the other hand KWS systems, that are used in practical applications, must run very energy efficiently on low energy devices, such as mobile phones, and give immediate and accurate responses to the users. 
A good elaboration on the requirements of KWS systems can be found in the motivation section of \cite{Warden2018}.