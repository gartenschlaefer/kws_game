% --
% Intro of keyword spotting

\section{The Keyword Spotting Task}\label{sec:intro_kws}
As described in \rsec{intro}, KWS is the task of classifying speech signals of spoken words to single keywords out of a set of keywords.
The set of keywords $S$, also referred to as vocabulary, can be defined by
\begin{equation}\label{eq:intro_kws_dict}
	S \coloneqq \{s_i \mid i = 0, 1, \dots, L\},
\end{equation}
where $s_i$ represents the $i$-th keyword of a set of $L$ keywords.
The task is to select the keyword closest to the spoken word from the user, denoted as target $t$.
The target does not necessarily have to be a member of the set of keywords $S$, in fact it can be any arbitrary word.
With the abstract formulation
\begin{equation}\label{eq:intro_kws_task}
	\hat{s} = \underset{s_i \in S}{\arg \min} \, \mathcal{D}(t, s_i),
\end{equation}
the most probable keyword $\hat{s}$ can be predicted, where $\mathcal{D}$ is some kind of distance measure between two words.
The formulation in \req{intro_kws_task} is merely semantic but KWS in computer systems have to cope with various transformations of raw input samples $\bm{x} \in \R^n$ of audio data with a total number of $n$ samples.
An inference from audio data to output class probabilities $\bm{y} \in \R^L$ can, for example, be achieved by the use of a neural network containing a softmax function at its last layer (transforming the output of the last layer to probability values).
The most probable keyword can therefore be picked by
\begin{equation}\label{eq:intro_kws_class}
	\hat{s} = \{s_i \mid \underset{i = 0, 1, \dots, L}{\arg \max} \, y_i\},
\end{equation}
where the index $i$ of the output class probability $y_i$ corresponds to the intended keyword $s_i$ in the vocabulary.

In comparison to full Automatic Speech Recognition (ASR), where whole sentences need to be identified, KWS operates merely on the word level.
Therefore, KWS is slightly easier to deploy and less complex than ASR.
Conversely, KWS systems have to run very energy efficiently on low-resource devices, such as mobile phones, and provide immediate and accurate responses to the users.
A good elaboration on the requirements of KWS systems can be found in the motivation section of \cite{Warden2018}.