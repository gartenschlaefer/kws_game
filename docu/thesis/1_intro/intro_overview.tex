% --
% intro overview of thesis

\section{Overview}\label{sec:intro_overview}
After this introduction, \rsec{back} provides background information of the individual disciplines within this thesis, further some research questions are listed and give insight into problems of the KWS task for video games.
\rsec{prev} lists previous and related work, where a small history about neural networks is given and important works on neural network architectures are referenced, as well as works that influenced and motivated this thesis.
The audio signal processing part is located in \rsec{signal} and explains the extraction of meaningful features for speech recognition, such as MFCCs and an onset detection method for KWS.
The feature extraction and onset detection is guided with examples to visualize their properties.
In \rsec{nn} theory about neural networks in general, as well as the used neural networks architectures and the adversarial pre-training are described.
The experiments are presented in \rsec{exp} with a thorough observation of the dataset and a general listing of experimental details beforehand. 
Experiments were conducted on the feature selection of MFCCs, the adversarial pre-training, the Wavenet architecture and the final experiments running on the whole dataset.
\rsec{game} describes the integrated KWS system in detail and presents the game design of the deployed video game.
The thesis finishes with a conclusion and an outlook to future work in \rsec{conclusion}.