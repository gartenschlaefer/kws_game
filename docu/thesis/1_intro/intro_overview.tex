% --
% intro overview of thesis

\section{Overview}\label{sec:intro_overview}
\thesisStateRevised
%This thesis is organized such, that each chapter leads to the next one and guiding someone who is interested in creating her own KWS game to follow through the process and challenges that might appear.
This thesis is organized such that each chapter connects to the next one in a typical processing pipeline for a KWS video game.
% prev
After this introduction, \rsec{prev} provides information about previous and related work.
A small history about neural networks is given and important works on neural network architectures used in this thesis are referrenced.
Further some works are presented that influenced and motivated this thesis.
%Others works provide benchmarks on the used speech commands dataset or describe neural networks models that perform well on speech.
%In any way it is important to know, what is happening in the field of KWS and in which direction research is heading towards.
%Some works are used as motivation and some to get an idea.
% signal
\rsec{signal} provides information about the audio signals processing pipeline and the extraction of meaningful features for speech recognition, such as Mel Frequency Cepstral Coefficients (MFCC).
The feature extraction is guided with examples to visualize their properties.
% neural networks
The used neural network architectures are described in detail in \rsec{nn}. 
Further some theory of neural networks in general, CNNs, GANs and Wavenets is provided and highlighted with training results from experiments.
% experiments
In \rsec{exp} information about the dataset and specific feature extraction is given as well as the experiments that are presented.
The experiments are done on the feature selection of MFCCs for the CNN models to determine the best suitable feature constellation the further experiments.
The adversarial pre-training of weights is compared to usual training of CNNs and Wavenet results are examined in contrast to the CNN based architectures.
% game
\rsec{game} describes the online and classification scheme in a potential video game application.
Further some game design ideas are presented and notes and challenges are stated regarding a KWS video game.
% conclusion
The thesis finishes with the conclusion and an outlook to future work in \rsec{conclusion}.

