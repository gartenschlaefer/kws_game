% --
% intro overview of thesis

\section{Overview}\label{sec:intro_overview}
After this introduction, \rsec{back} provides background information of the individual disciplines within this thesis.
Furthermore, it lists the research questions and give insight into problems of the KWS task for video games.
\rsec{prev} concentrates on previous and related scientific work, where a small history about neural networks is provided as well as important work on neural network architectures and work that influenced and motivated this thesis.
The audio signal processing part is located in \rsec{signal} and explains the extraction of meaningful features for speech recognition, such as MFCCs and an onset detection method for KWS.
The feature extraction and onset detection is guided with examples to visualize their properties.
In \rsec{nn}, general theory about neural networks is described as well as the used neural networks architectures and the adversarial pre-training.
\rsec{exp} presents the experiments and further includes a thorough observation of the dataset and a general listing of experimental details beforehand.
The experiments were conducted on the feature selection of MFCCs, the adversarial pre-training, the Wavenet architecture, and the final experiments on the whole dataset.
\rsec{game} describes the integrated KWS system in detail and presents the game design of the deployed video game.
This thesis finishes in \rsec{conclusion} with a conclusion and an outlook to future work.