% --
% intro overview of thesis

\section{Overview}\label{sec:intro_overview}
\thesisStateRevised
This thesis is organized such that each chapter connects to the next one in a typical processing pipeline for a KWS video game.
After this introduction, \rsec{back} provides further information of the individual disciplines of the KWS task applied in video games, futher some research questions are listed and give insight into problems of this task.
\rsec{prev} is about previous and related work, where a small history about neural networks is given and important works on neural network architectures are referrenced, as well as works that influenced and motivated this thesis.
\rsec{signal} provides information about the audio signals processing pipeline and the extraction of meaningful features for speech recognition, such as Mel Frequency Cepstral Coefficients (MFCC).
The feature extraction is guided with examples to visualize their properties.
The used neural network architectures are described in detail in \rsec{nn}. 
Further some theory about neural networks in general, CNNs, GANs and Wavenets is provided and highlighted with training results from experiments.
In \rsec{exp} information about the dataset and specific feature extraction is given as well as the experiments that are presented.
The experiments were done on the feature selection of MFCCs for the CNN models to determine the best suitable feature constellation for further experiments.
The adversarial pre-training of weights is compared to usual training of CNNs and Wavenet results are examined in contrast to the CNN based architectures.
\rsec{game} describes the online and classification scheme in a potential video game application.
Further some game design ideas are presented and notes and challenges are stated regarding a KWS video game.
The thesis finishes with the conclusion and an outlook to future work in \rsec{conclusion}.

