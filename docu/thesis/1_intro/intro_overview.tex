% --
% intro overview of thesis

\section{Overview of this thesis and notations}\label{sec:intro_overview}

This thesis is organized such, that each chapter leads to the next one and guiding someone who is interested in creating her own KWS game to follow through the process and challenges that might appear.
After this introduction, \rsec{prev} provides information about previous and related work.
Some history about neural networks is given and the most important neural network achievements regarding this thesis, are explained in their essential structures.
Further some works are presented that influenced and motivated this thesis and
others provide benchmarks on the speech commands dataset used or describe neural networks models that perform well on speech.
In any way it is important to know, what is happening in the field of KWS and in which direction research is heading towards.
%Some works are used as motivation and some to get an idea.

% visual guidance
\subsection{Visual Guidance}\label{sec:intro_overview_visual}
To get a better overview on the presentation of data and results, context specific color color-schemes are used within this thesis.
% There exist following context abstractions:

% \begin{itemize}
%     \item raw waveforms from soundfiles
%     \item extracted features, e.g. MFCCs
%     \item weights matrices of neural network models
%     \item training scores
% \end{itemize}

% ipa
\subsection{International Phonetic Alphabet}\label{sec:intro_overview_ipa}
The International Phonetic Alphabet is, as it name suggest, an alphabet of phonetics, where phonetics can be defined as human speaking sounds.
A word formed with letters from a real Alphabet might trick the speaker of this word on how to pronounce it correctly, but described with the IPA, there is no misconception.
In some sections within this thesis, there are plots with phonetic transcriptions containing IPA characters, some special ones are described in \rtab{intro_overview_ipa}.

% ipa table
\begin{table}[ht!]
\begin{center}
\caption{Some IPA and silence symbol with description.}
\begin{tabular}{ M{2cm}  M{9cm} }
\toprule
\textbf{IPA Symbol} & \textbf{Meaning} \\
\midrule
\textturnv & back vowel: \enquote{A}, open-mid roundend mouth \\
\textupsilon & back vowel: between \enquote{O} and \enquote{U}, nearly closed rounded mouth\\
\textinvglotstop & glottal stop\\
\midrule
sil & silence, no ipa symbol!\\
\bottomrule
\label{tab:intro_overview_ipa}
\end{tabular}
\end{center}
\end{table}
\FloatBarrier
\noindent



% math
\subsection{Mathematical Notations}\label{sec:intro_overview_math}
The mathematical equations or expressions contained in this thesis are following some criteria.
Vectors and scalars are usually written in small letters with no special indication for vectors, such as making them bold.
Matrices are usually written in capital letters.
The dimension of vectors and matrices are provided, such as $x\in\R^n$ or $X\in\C^{m \times n}$, or follow from the context.
Other than that